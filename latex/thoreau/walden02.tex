% Adapted from http://thoreau.eserver.org/walden02.html

\documentclass[letterpaper,12pt]{article}
\usepackage[margin=0.977in]{geometry}
\usepackage{csquotes}

\title{Walden: Where I Lived, and What I Lived for}
\author{Henry David Thoreau}
\date{1849}

\renewcommand{\theparagraph}{[\arabic{paragraph}]}
\setcounter{secnumdepth}{4}

\usepackage{scrextend}
\deffootnote[1em]{1.5em}{1em}{\textsuperscript{\thefootnotemark}}

\begin{document}

\maketitle

\paragraph{} At a certain season of our life we are accustomed to consider every
spot as the possible site of a house. I have thus surveyed the country on every
side within a dozen miles of where I live. In imagination I have bought all the
farms in succession, for all were to be bought, and I knew their price. I walked
over each farmer's premises, tasted his wild apples, discoursed on husbandry
with him, took his farm at his price, at any price, mortgaging it to him in my
mind; even put a higher price on it --- took everything but a deed of it ---
took his word for his deed, for I dearly love to talk --- cultivated it, and him
too to some extent, I trust, and withdrew when I had enjoyed it long enough,
leaving him to carry it on.\footnote{E.B. White wrote of this sentence:
    \enquote{A copy-desk man would get a double hernia trying to clean up that
        sentence for the management, but the sentence needs no fixing, for it
        perfectly captures the meaning of the writer and the quality of the
        ramble.}} This experience entitled me to be regarded as a sort of
real-estate broker by my friends. Wherever I sat, there I might live, and the
landscape radiated from me accordingly. What is a house but a sedes, a seat? ---
better if a country seat.  I discovered many a site for a house not likely to be
soon improved, which some might have thought too far from the village, but to my
eyes the village was too far from it. Well, there I might live, I said; and
there I did live, for an hour, a summer and a winter life; saw how I could let
the years run off, buffet the winter through, and see the spring come in. The
future inhabitants of this region, wherever they may place their houses, may be
sure that they have been anticipated. An afternoon sufficed to lay out the land
into orchard, wood-lot, and pasture, and to decide what fine oaks or pines
should be left to stand before the door, and whence each blasted tree could be
seen to the best advantage; and then I let it lie, fallow, perchance, for a man
is rich in proportion to the number of things which he can afford to let alone.

\paragraph{} My imagination carried me so far that I even had the refusal of
several farms --- the refusal was all I wanted --- but I never got my fingers
burned by actual possession. The nearest that I came to actual possession was
when I bought the Hollowell place, and had begun to sort my seeds, and collected
materials with which to make a wheelbarrow to carry it on or off with; but
before the owner gave me a deed of it, his wife --- every man has such a wife
--- changed her mind and wished to keep it, and he offered me ten dollars to
release him. Now, to speak the truth, I had but ten cents in the world, and it
surpassed my arithmetic to tell, if I was that man who had ten cents, or who had
a farm, or ten dollars, or all together. However, I let him keep the ten dollars
and the farm too, for I had carried it far enough; or rather, to be generous,
I sold him the farm for just what I gave for it, and, as he was not a rich man,
made him a present of ten dollars, and still had my ten cents, and seeds, and
materials for a wheelbarrow left. I found thus that I had been a rich man
without any damage to my poverty. But I retained the landscape, and I have since
annually carried off what it yielded without a wheelbarrow. With respect to
landscapes,

\begin{verse}
    \enquote{I am monarch of all I survey, \\
        My right there is none to dispute.}\footnote{William Cowper (1731--1800)
        English poet, hymnist, \textit{The Solitude of Alexander Selkirk}}
\end{verse}

\paragraph{} I have frequently seen a poet withdraw, having enjoyed the most
valuable part of a farm, while the crusty farmer supposed that he had got a few
wild apples only. Why, the owner does not know it for many years when a poet has
put his farm in rhyme, the most admirable kind of invisible fence, has fairly
impounded it, milked it, skimmed it, and got all the cream, and left the farmer
only the skimmed milk.

\paragraph{} The real attractions of the Hollowell farm, to me, were: its
complete retirement, being, about two miles from the village, half a mile from
the nearest neighbor, and separated from the highway by a broad field; its
bounding on the river, which the owner said protected it by its fogs from frosts
in the spring, though that was nothing to me; the gray color and ruinous state
of the house and barn, and the dilapidated fences, which put such an interval
between me and the last occupant; the hollow and lichen-covered apple trees,
gnawed by rabbits, showing what kind of neighbors I should have; but above all,
the recollection I had of it from my earliest voyages up the river, when the
house was concealed behind a dense grove of red maples, through which I heard
the house-dog bark. I was in haste to buy it, before the proprietor finished
getting out some rocks, cutting down the hollow apple trees, and grubbing up
some young birches which had sprung up in the pasture, or, in short, had made
any more of his improvements. To enjoy these advantages I was ready to carry it
on; like Atlas,\footnote{in Greek mythology Atlas supported the heavens on his
    shoulders} to take the world on my shoulders --- I never heard what
compensation he received for that --- and do all those things which had no other
motive or excuse but that I might pay for it and be unmolested in my possession
of it; for I knew all the while that it would yield the most abundant crop of
the kind I wanted, if I could only afford to let it alone. But it turned out as
I have said.

\paragraph{} All that I could say, then, with respect to farming on a large
scale --- I have always cultivated a garden --- was, that I had had my seeds
ready. Many think that seeds improve with age. I have no doubt that time
discriminates between the good and the bad; and when at last I shall plant,
I shall be less likely to be disappointed. But I would say to my fellows, once
for all, As long as possible live free and uncommitted. It makes but little
difference whether you are committed to a farm or the county jail.

\paragraph{} Old Cato\footnote{Marcus Porcius Cato (234--149 B.C.) Roman
    agricultural author}, whose \enquote{De Re Rustic\^a}\footnote{\textit{De Re
        Rustica (Agriculture)}, by Roman authors Marcus Porcius Cato (234--149
    BC) and Marcus Terentius Varro (116 BC-27 BC)} is my
\enquote{Cultivator,}\footnote{publications of Thoreau's time, such as the
    \textit{Boston Cultivator} or the \textit{New England Cultivator}} says ---
and the only translation I have seen makes sheer nonsense of the passage ---
\enquote{When you think of getting a farm turn it thus in your mind, not to buy
    greedily; nor spare your pains to look at it, and do not think it enough to
    go round it once. The oftener you go there the more it will please you, if
    it is good.} I think I shall not buy greedily, but go round and round it as
long as I live, and be buried in it first, that it may please me the more at
last.


\paragraph{} The present was my next experiment of this kind, which I purpose to
describe more at length, for convenience putting the experience of two years
into one. As I have said, I do not propose to write an ode to dejection, but to
brag as lustily as chanticleer in the morning, standing on his roost, if only to
wake my neighbors up.


\paragraph{} When first I took up my abode in the woods, that is, began to spend
my nights as well as days there, which, by accident, was on Independence Day, or
the Fourth of July, 1845, my house was not finished for winter, but was merely
a defense against the rain, without plastering or chimney, the walls being of
rough, weather-stained boards, with wide chinks, which made it cool at night.
The upright white hewn studs and freshly planed door and window casings gave it
a clean and airy look, especially in the morning, when its timbers were
saturated with dew, so that I fancied that by noon some sweet gum would exude
from them. To my imagination it retained throughout the day more or less of this
auroral character, reminding me of a certain house on a mountain which I had
visited a year before. This was an airy and unplastered cabin, fit to entertain
a traveling god, and where a goddess might trail her garments. The winds which
passed over my dwelling were such as sweep over the ridges of mountains, bearing
the broken strains, or celestial parts only, of terrestrial music. The morning
wind forever blows, the poem of creation is uninterrupted; but few are the ears
that hear it. Olympus\footnote{in Greek mythology, home of the gods} is but the
outside of the earth everywhere.

\paragraph{} The only house I had been the owner of before, if I except a boat,
was a tent, which I used occasionally when making excursions in the summer, and
this is still rolled up in my garret; but the boat, after passing from hand to
hand, has gone down the stream of time. With this more substantial shelter about
me, I had made some progress toward settling in the world. This frame, so
slightly clad, was a sort of crystallization around me, and reacted on the
builder. It was suggestive somewhat as a picture in outlines. I did not need to
go outdoors to take the air, for the atmosphere within had lost none of its
freshness. It was not so much within doors as behind a door where I sat, even in
the rainiest weather. The Harivansa\footnote{5th century Hindu epic poem} says,
\enquote{An abode without birds is like a meat without seasoning.} Such was not
my abode, for I found myself suddenly neighbor to the birds; not by having
imprisoned one, but having caged myself near them. I was not only nearer to some
of those which commonly frequent the garden and the orchard, but to those
smaller and more thrilling songsters of the forest which never, or rarely,
serenade a villager --- the wood thrush, the veery, the scarlet tanager, the
field sparrow, the whip-poor-will, and many others.

\paragraph{} I was seated by the shore of a small pond, about a mile and a half
south of the village of Concord and somewhat higher than it, in the midst of an
extensive wood between that town and Lincoln, and about two miles south of that
our only field known to fame, Concord Battle Ground; but I was so low in the
woods that the opposite shore, half a mile off, like the rest, covered with
wood, was my most distant horizon. For the first week, whenever I looked out on
the pond it impressed me like a tarn high up on the side of a mountain, its
bottom far above the surface of other lakes, and, as the sun arose, I saw it
throwing off its nightly clothing of mist, and here and there, by degrees, its
soft ripples or its smooth reflecting surface was revealed, while the mists,
like ghosts, were stealthily withdrawing in every direction into the woods, as
at the breaking up of some nocturnal conventicle. The very dew seemed to hang
upon the trees later into the day than usual, as on the sides of mountains.

\paragraph{} This small lake was of most value as a neighbor in the intervals of
a gentle rain-storm in August, when, both air and water being perfectly still,
but the sky overcast, mid-afternoon had all the serenity of evening, and the
wood thrush sang around, and was heard from shore to shore. A lake like this is
never smoother than at such a time; and the clear portion of the air above it
being, shallow and darkened by clouds, the water, full of light and reflections,
becomes a lower heaven itself so much the more important. From a hill-top near
by, where the wood had been recently cut off, there was a pleasing vista
southward across the pond, through a wide indentation in the hills which form
the shore there, where their opposite sides sloping toward each other suggested
a stream flowing out in that direction through a wooded valley, but stream there
was none. That way I looked between and over the near green hills to some
distant and higher ones in the horizon, tinged with blue. Indeed, by standing on
tiptoe I could catch a glimpse of some of the peaks of the still bluer and more
distant mountain ranges in the northwest, those true-blue coins from heaven's
own mint, and also of some portion of the village. But in other directions, even
from this point, I could not see over or beyond the woods which surrounded me.
It is well to have some water in your neighborhood, to give buoyancy to and
float the earth. One value even of the smallest well is, that when you look into
it you see that earth is not continent but insular. This is as important as that
it keeps butter cool. When I looked across the pond from this peak toward the
Sudbury meadows, which in time of flood I distinguished elevated perhaps by
a mirage in their seething valley, like a coin in a basin, all the earth beyond
the pond appeared like a thin crust insulated and floated even by this small
sheet of interverting water, and I was reminded that this on which I dwelt was
but dry land.

\paragraph{} Though the view from my door was still more contracted, I did not
feel crowded or confined in the least. There was pasture enough for my
imagination. The low shrub oak plateau to which the opposite shore arose
stretched away toward the prairies of the West and the steppes of Tartary,
affording ample room for all the roving families of men. \enquote{There are none
    happy in the world but beings who enjoy freely a vast horizon} --- said
Damodara,\footnote{another name for the Hindu god Krishna} when his herds
required new and larger pastures.

\paragraph{} Both place and time were changed, and I dwelt nearer to those parts
of the universe and to those eras in history which had most attracted me. Where
I lived was as far off as many a region viewed nightly by astronomers. We are
wont to imagine rare and delectable places in some remote and more celestial
corner of the system, behind the constellation of Cassiopeia's Chair, far from
noise and disturbance. I discovered that my house actually had its site in such
a withdrawn, but forever new and unprofaned, part of the universe. If it were
worth the while to settle in those parts near to the Pleiades or the Hyades, to
Aldebaran or Altair,\footnote{Cassiopeia's Chair, Pleiades, and Hyades are
    constellations, Aldebaran and Altair are stars}  then I was really there, or
at an equal remoteness from the life which I had left behind, dwindled and
twinkling with as fine a ray to my nearest neighbor, and to be seen only in
moonless nights by him. Such was that part of creation where I had squatted, ---

\begin{verse}
    \enquote{There was a shepherd that did live, \\
        And held his thoughts as high \\
        As were the mounts whereon his flocks \\
        Did hourly feed him by.}\footnote{anonymous, published 1610}
\end{verse}

What should we think of the shepherd's life if his flocks always wandered to
higher pastures than his thoughts?

\paragraph{} Every morning was a cheerful invitation to make my life of equal
simplicity, and I may say innocence, with Nature herself. I have been as sincere
a worshiper of Aurora\footnote{in Roman mythology, the goddess of dawn} as the
Greeks. I got up early and bathed in the pond; that was a religious exercise,
and one of the best things which I did.  They say that characters were engraven
on the bathing tub of King Tching Thang\footnote{another name for Confucius} to
this effect: \enquote{Renew thyself completely each day; do it again, and again,
    and forever again.} I can understand that. Morning brings back the heroic
ages.  I was as much affected by the faint hum of a mosquito making its
invisible and unimaginable tour through my apartment at earliest dawn, when
I was sitting with door and windows open, as I could be by any trumpet that ever
sang of fame. It was Homer's requiem; itself an Iliad and
Odyssey\footnote{\textit{Iliad} and \textit{Odyssey}, attributed to Homer, 8th
    cent.\ B.C. Greek epic poet} in the air, singing its own wrath and
wanderings. There was something cosmical about it; a standing advertisement,
till forbidden, of the everlasting vigor and fertility of the world. The
morning, which is the most memorable season of the day, is the awakening hour.
Then there is least somnolence in us; and for an hour, at least, some part of us
awakes which slumbers all the rest of the day and night. Little is to be
expected of that day, if it can be called a day, to which we are not awakened by
our Genius, but by the mechanical nudgings of some servitor, are not awakened by
our own newly acquired force and aspirations from within, accompanied by the
undulations of celestial music, instead of factory bells, and a fragrance
filling the air --- to a higher life than we fell asleep from; and thus the
darkness bear its fruit, and prove itself to be good, no less than the light.
That man who does not believe that each day contains an earlier, more sacred,
and auroral hour than he has yet profaned, has despaired of life, and is
pursuing a descending and darkening way. After a partial cessation of his
sensuous life, the soul of man, or its organs rather, are reinvigorated each
day, and his Genius tries again what noble life it can make. All memorable
events, I should say, transpire in morning time and in a morning atmosphere. The
Vedas\footnote{Brahmin religious books} say, \enquote{All intelligences awake
    with the morning.} Poetry and art, and the fairest and most memorable of the
actions of men, date from such an hour.  All poets and heroes, like
Memnon,\footnote{statue in ancient Egypt said to produce music at dawn} are the
children of Aurora, and emit their music at sunrise. To him whose elastic and
vigorous thought keeps pace with the sun, the day is a perpetual morning. It
matters not what the clocks say or the attitudes and labors of men. Morning is
when I am awake and there is a dawn in me. Moral reform is the effort to throw
off sleep. Why is it that men give so poor an account of their day if they have
not been slumbering? They are not such poor calculators. If they had not been
overcome with drowsiness, they would have performed something. The millions are
awake enough for physical labor; but only one in a million is awake enough for
effective intellectual exertion, only one in a hundred millions to a poetic or
divine life. To be awake is to be alive. I have never yet met a man who was
quite awake. How could I have looked him in the face?

\paragraph{} We must learn to reawaken and keep ourselves awake, not by
mechanical aids, but by an infinite expectation of the dawn, which does not
forsake us in our soundest sleep. I know of no more encouraging fact than the
unquestionable ability of man to elevate his life by a conscious endeavor. It is
something to be able to paint a particular picture, or to carve a statue, and so
to make a few objects beautiful; but it is far more glorious to carve and paint
the very atmosphere and medium through which we look, which morally we can do.
To affect the quality of the day, that is the highest of arts. Every man is
tasked to make his life, even in its details, worthy of the contemplation of his
most elevated and critical hour. If we refused, or rather used up, such paltry
information as we get, the oracles would distinctly inform us how this might be
done.

\paragraph{} I went to the woods because I wished to live deliberately, to front
only the essential facts of life, and see if I could not learn what it had to
teach, and not, when I came to die, discover that I had not lived. I did not
wish to live what was not life, living is so dear; nor did I wish to practice
resignation, unless it was quite necessary. I wanted to live deep and suck out
all the marrow of life, to live so sturdily and Spartan-like as to put to rout
all that was not life, to cut a broad swath and shave close, to drive life into
a corner, and reduce it to its lowest terms, and, if it proved to be mean, why
then to get the whole and genuine meanness of it, and publish its meanness to
the world; or if it were sublime, to know it by experience, and be able to give
a true account of it in my next excursion. For most men, it appears to me, are
in a strange uncertainty about it, whether it is of the devil or of God, and
have somewhat hastily concluded that it is the chief end of man here to
\enquote{glorify God and enjoy him forever.}\footnote{Westminster Catechism}

\paragraph{} Still we live meanly, like ants; though the fable tells us that we
were long ago changed into men; like pygmies we fight with cranes; it is error
upon error, and clout upon clout, and our best virtue has for its occasion
a superfluous and evitable wretchedness. Our life is frittered away by detail.
An honest man has hardly need to count more than his ten fingers, or in extreme
cases he may add his ten toes, and lump the rest. Simplicity, simplicity,
simplicity! I say, let your affairs be as two or three, and not a hundred or
a thousand; instead of a million count half a dozen, and keep your accounts on
your thumb-nail. In the midst of this chopping sea of civilized life, such are
the clouds and storms and quicksands and thousand-and-one items to be allowed
for, that a man has to live, if he would not founder and go to the bottom and
not make his port at all, by dead reckoning, and he must be a great calculator
indeed who succeeds. Simplify, simplify. Instead of three meals a day, if it be
necessary eat but one; instead of a hundred dishes, five; and reduce other
things in proportion. Our life is like a German Confederacy,\footnote{group of
    European states, 1815--1866} made up of petty states, with its boundary
forever fluctuating, so that even a German cannot tell you how it is bounded at
any moment. The nation itself, with all its so-called internal improvements,
which, by the way are all external and superficial, is just such an unwieldy and
overgrown establishment, cluttered with furniture and tripped up by its own
traps, ruined by luxury and heedless expense, by want of calculation and
a worthy aim, as the million households in the land; and the only cure for it,
as for them, is in a rigid economy, a stern and more than Spartan\footnote{like
    the Spartans of ancient Greece, disciplined, austere} simplicity of life and
elevation of purpose. It lives too fast. Men think that it is essential that the
Nation have commerce, and export ice, and talk through a telegraph, and ride
thirty miles an hour, without a doubt, whether they do or not; but whether we
should live like baboons or like men, is a little uncertain. If we do not get
out sleepers,\footnote{wooden railroad ties that support the rails} and forge
rails, and devote days and nights to the work, but go to tinkering upon our
lives to improve them, who will build railroads? And if railroads are not built,
how shall we get to heaven in season? But if we stay at home and mind our
business, who will want railroads? We do not ride on the railroad; it rides upon
us. Did you ever think what those sleepers are that underlie the railroad? Each
one is a man, an Irishman, or a Yankee man. The rails are laid on them, and they
are covered with sand, and the cars run smoothly over them. They are sound
sleepers, I assure you. And every few years a new lot is laid down and run over;
so that, if some have the pleasure of riding on a rail, others have the
misfortune to be ridden upon. And when they run over a man that is walking in
his sleep, a supernumerary sleeper in the wrong position, and wake him up, they
suddenly stop the cars, and make a hue and cry about it, as if this were an
exception.  I am glad to know that it takes a gang of men for every five miles
to keep the sleepers down and level in their beds as it is, for this is a sign
that they may sometime get up again.

\paragraph{} Why should we live with such hurry and waste of life? We are
determined to be starved before we are hungry. Men say that a stitch in time
saves nine, and so they take a thousand stitches today to save nine tomorrow. As
for work, we haven't any of any consequence. We have the Saint Vitus'
dance,\footnote{chorea, a nervous disorder characterized by involuntary
    movements} and cannot possibly keep our heads still. If I should only give
a few pulls at the parish bell-rope, as for a fire, that is, without setting the
bell, there is hardly a man on his farm in the outskirts of Concord,
notwithstanding that press of engagements which was his excuse so many times
this morning, nor a boy, nor a woman, I might almost say, but would forsake all
and follow that sound, not mainly to save property from the flames, but, if we
will confess the truth, much more to see it burn, since burn it must, and we, be
it known, did not set it on fire --- or to see it put out, and have a hand in
it, if that is done as handsomely; yes, even if it were the parish church
itself. Hardly a man takes a half-hour's nap after dinner, but when he wakes he
holds up his head and asks, \enquote{What's the news?} as if the rest of mankind
had stood his sentinels. Some give directions to be waked every half-hour,
doubtless for no other purpose; and then, to pay for it, they tell what they
have dreamed. After a night's sleep the news is as indispensable as the
breakfast. \enquote{Pray tell me anything new that has happened to a man
    anywhere on this globe} --- and he reads it over his coffee and rolls, that
a man has had his eyes gouged out this morning on the Wachito
River;\footnote{river in Arkansas and Louisiana} never dreaming the while that
he lives in the dark unfathomed mammoth cave of this world, and has but the
rudiment of an eye himself.

\paragraph{} For my part, I could easily do without the post-office. I think
that there are very few important communications made through it. To speak
critically, I never received more than one or two letters in my life --- I wrote
this some years ago --- that were worth the postage. The penny-post is,
commonly, an institution through which you seriously offer a man that penny for
his thoughts which is so often safely offered in jest. And I am sure that
I never read any memorable news in a newspaper. If we read of one man robbed, or
murdered, or killed by accident, or one house burned, or one vessel wrecked, or
one steamboat blown up, or one cow run over on the Western Railroad, or one mad
dog killed, or one lot of grasshoppers in the winter --- we never need read of
another. One is enough. If you are acquainted with the principle, what do you
care for a myriad instances and applications? To a philosopher all news, as it
is called, is gossip, and they who edit and read it are old women over their
tea. Yet not a few are greedy after this gossip. There was such a rush, as
I hear, the other day at one of the offices to learn the foreign news by the
last arrival, that several large squares of plate glass belonging to the
establishment were broken by the pressure --- news which I seriously think
a ready wit might write a twelve-month, or twelve years, beforehand with
sufficient accuracy. As for Spain, for instance, if you know how to throw in Don
Carlos and the Infanta, and Don Pedro and Seville and Granada,\footnote{relating
    to Spanish \& Portuguese politics, 1830's \& 1840's} from time to time in
the right proportions --- they may have changed the names a little since I saw
the papers --- and serve up a bull-fight when other entertainments fail, it will
be true to the letter, and give us as good an idea of the exact state or ruin of
things in Spain as the most succinct and lucid reports under this head in the
newspapers: and as for England, almost the last significant scrap of news from
that quarter was the revolution of 1649; and if you have learned the history of
her crops for an average year, you never need attend to that thing again, unless
your speculations are of a merely pecuniary character. If one may judge who
rarely looks into the newspapers, nothing new does ever happen in foreign parts,
a French revolution not excepted.

\paragraph{} What news! how much more important to know what that is which was
never old!  \enquote{Kieou-pe-yu\footnote{character in a book by Confucius}
    (great dignitary of the state of Wei) sent a man to Khoung-tseu to know his
    news. Khoung-tseu caused the messenger to be seated near him, and questioned
    him in these terms: What is your master doing? The messenger answered with
    respect: My master desires to diminish the number of his faults, but he
    cannot accomplish it. The messenger being gone, the philosopher remarked:
    What a worthy messenger! What a worthy messenger!} The preacher, instead of
vexing the ears of drowsy farmers on their day of rest at the end of the week
--- for Sunday is the fit conclusion of an ill-spent week, and not the fresh and
brave beginning of a new one --- with this one other draggle-tail of a sermon,
should shout with thundering voice, \enquote{Pause! Avast! Why so seeming fast,
    but deadly slow?}

\paragraph{} Shams and delusions are esteemed for soundest truths, while reality
is fabulous. If men would steadily observe realities only, and not allow
themselves to be deluded, life, to compare it with such things as we know, would
be like a fairy tale and the Arabian Nights' Entertainments. If we respected
only what is inevitable and has a right to be, music and poetry would resound
along the streets. When we are unhurried and wise, we perceive that only great
and worthy things have any permanent and absolute existence, that petty fears
and petty pleasures are but the shadow of the reality. This is always
exhilarating and sublime. By closing the eyes and slumbering, and consenting to
be deceived by shows, men establish and confirm their daily life of routine and
habit everywhere, which still is built on purely illusory foundations. Children,
who play life, discern its true law and relations more clearly than men, who
fail to live it worthily, but who think that they are wiser by experience, that
is, by failure. I have read in a Hindoo book, that \enquote{there was a king's
    son, who, being expelled in infancy from his native city, was brought up by
    a forester, and, growing up to maturity in that state, imagined himself to
    belong to the barbarous race with which he lived. One of his father's
    ministers having discovered him, revealed to him what he was, and the
    misconception of his character was removed, and he knew himself to be
    a prince. So soul,} continues the Hindoo philosopher, \enquote{from the
    circumstances in which it is placed, mistakes its own character, until the
    truth is revealed to it by some holy teacher, and then it knows itself to be
    Brahme.}\footnote{Brahma, Hindu god of creation} I perceive that we
inhabitants of New England live this mean life that we do because our vision
does not penetrate the surface of things. We think that that is which appears to
be. If a man should walk through this town and see only the reality, where,
think you, would the \enquote{Mill-dam} go to? If he should give us an account
of the realities he beheld there, we should not recognize the place in his
description. Look at a meeting-house, or a court-house, or a jail, or a shop, or
a dwelling-house, and say what that thing really is before a true gaze, and they
would all go to pieces in your account of them. Men esteem truth remote, in the
outskirts of the system, behind the farthest star, before Adam and after the
last man. In eternity there is indeed something true and sublime. But all these
times and places and occasions are now and here. God himself culminates in the
present moment, and will never be more divine in the lapse of all the ages. And
we are enabled to apprehend at all what is sublime and noble only by the
perpetual instilling and drenching of the reality that surrounds us. The
universe constantly and obediently answers to our conceptions; whether we travel
fast or slow, the track is laid for us. Let us spend our lives in conceiving
then. The poet or the artist never yet had so fair and noble a design but some
of his posterity at least could accomplish it.

\paragraph{} Let us spend one day as deliberately as Nature, and not be thrown
off the track by every nutshell and mosquito's wing that falls on the rails. Let
us rise early and fast, or break fast, gently and without perturbation; let
company come and let company go, let the bells ring and the children cry ---
determined to make a day of it. Why should we knock under and go with the
stream? Let us not be upset and overwhelmed in that terrible rapid and whirlpool
called a dinner, situated in the meridian shallows. Weather this danger and you
are safe, for the rest of the way is down hill. With unrelaxed nerves, with
morning vigor, sail by it, looking another way, tied to the mast like
Ulysses.\footnote{Roman name for Odysseus, character in Homer's Iliad and
    Odyssey} If the engine whistles, let it whistle till it is hoarse for its
pains. If the bell rings, why should we run? We will consider what kind of music
they are like. Let us settle ourselves, and work and wedge our feet downward
through the mud and slush of opinion, and prejudice, and tradition, and
delusion, and appearance, that alluvion which covers the globe, through Paris
and London, through New York and Boston and Concord, through Church and State,
through poetry and philosophy and religion, till we come to a hard bottom and
rocks in place, which we can call reality, and say, This is, and no mistake; and
then begin, having a point d'appui,\footnote{a point of support} below freshet
and frost and fire, a place where you might found a wall or a state, or set
a lamp-post safely, or perhaps a gauge, not a Nilometer,\footnote{gauge used to
    measure the rise of the Nile River in Egypt} but a Realometer, that future
ages might know how deep a freshet of shams and appearances had gathered from
time to time. If you stand right fronting and face to face to a fact, you will
see the sun glimmer on both its surfaces, as if it were a cimeter,\footnote{also
    called a scimiter, a saber with a curved blade, used in the Middle East} and
feel its sweet edge dividing you through the heart and marrow, and so you will
happily conclude your mortal career. Be it life or death, we crave only reality.
If we are really dying, let us hear the rattle in our throats and feel cold in
the extremities; if we are alive, let us go about our business.

\paragraph{} Time is but the stream I go a-fishing in. I drink at it; but while
I drink I see the sandy bottom and detect how shallow it is. Its thin current
slides away, but eternity remains. I would drink deeper; fish in the sky, whose
bottom is pebbly with stars. I cannot count one. I know not the first letter of
the alphabet.  I have always been regretting that I was not as wise as the day
I was born. The intellect is a cleaver; it discerns and rifts its way into the
secret of things.  I do not wish to be any more busy with my hands than is
necessary. My head is hands and feet. I feel all my best faculties concentrated
in it. My instinct tells me that my head is an organ for burrowing, as some
creatures use their snout and fore paws, and with it I would mine and burrow my
way through these hills. I think that the richest vein is somewhere hereabouts;
so by the divining-rod and thin rising vapors I judge; and here I will begin to
mine.
\end{document}

