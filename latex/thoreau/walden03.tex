% Adapted from http://thoreau.eserver.org/walden03.html

\documentclass[letterpaper,12pt]{article}
\usepackage[margin=0.819in]{geometry}
\usepackage{csquotes}
\usepackage{epigraph}
\usepackage{scrextend}

\title{\vspace{-2em}Walden: Reading}
\author{Henry David Thoreau}
\date{1849}

\renewcommand{\theparagraph}{[\arabic{paragraph}]}
\setcounter{secnumdepth}{4}

\deffootnote[1em]{1.5em}{1em}{\textsuperscript{\thefootnotemark}}

\setlength{\epigraphwidth}{\textwidth}
\setlength\epigraphrule{0pt}
\renewcommand{\textflush}{flushepinormal}

\begin{document}
\maketitle

\hrule
\epigraph{\enquote{On one occasion he went to the [Harvard] University Library
        to procure some books.  The librarian refused to lend them.  Mr.\
        Thoreau repaired to the President, who stated to him the rules and
        usages, which permitted the loan of books to resident graduates, to
        clergymen who were alumni, and to some other residents within a circle
        of ten miles radius from the College.  Mr.\ Thoreau explained to the
        President that the railroad had destroyed the old scale of distances,
        --- that the library was useless, yes, and President and College
        useless, on the terms of his rules, --- that the one benefit he owed to
        the College was its library, --- that, at this moment, not only his want
        of books was imperative, but he wanted a large number of books, and
        assured him that he, Thoreau, and not the librarian, was the proper
        custodian of these.  In short, the President found the petitioner so
        formidable, and the rules getting to look so ridiculous, that he ended
        by giving him a privilege which in his hands proved unlimited
        thereafter.}}{---Ralph Waldo Emerson}
\hrule

\paragraph{}
With a little more deliberation in the choice of their pursuits,
all men would perhaps become essentially students and observers, for certainly
their nature and destiny are interesting to all alike.  In accumulating property
for ourselves or our posterity, in founding a family or a state, or acquiring
fame even, we are mortal; but in dealing with truth we are immortal, and need
fear no change nor accident.  The oldest Egyptian or Hindoo philosopher raised a
corner of the veil from the statue of the divinity; and still the trembling robe
remains raised, and I gaze upon as fresh a glory as he did, since it was I in
him that was then so bold, and it is he in me that now reviews the vision.  No
dust has settled on that robe; no time has elapsed since that divinity was
revealed. That time which we really improve, or which is improvable, is neither
past, present, nor future.

\paragraph{}
My residence was more favorable, not only to thought, but to serious reading,
than a university; and though I was beyond the range of the ordinary circulating
library, I had more than ever come within the influence of those books which
circulate round the world, whose sentences were first written on bark, and are
now merely copied from time to time on to linen paper.  Says the poet Mîr Camar
Uddin Mast,\footnote{Mir Camar Uddin Mast, 18th century Persian poet}
\enquote{Being seated, to run through the region of the spiritual world; I have
    had this advantage in books. To be intoxicated by a single glass of wine; I
    have experienced this pleasure when I have drunk the liquor of the esoteric
    doctrines.} I kept Homer's Iliad\footnote{\textit{Iliad}, attributed to
    Homer, 5th cent.\ B.C. Greek epic poet} on my table through the summer,
though I looked at his page only now and then.  Incessant labor with my hands,
at first, for I had my house to finish and my beans to hoe at the same time,
made more study impossible.  Yet I sustained myself by the prospect of such
reading in future.  I read one or two shallow books of travel in the intervals
of my work, till that employment made me ashamed of myself, and I asked where it
was then that I lived.

\paragraph{}
The student may read Homer or \AE{}schylus\footnote{\AE{}schylus (525--456 B.C.)
    Greek dramatist} in the Greek without danger of dissipation or
luxuriousness, for it implies that he in some measure emulate their heroes, and
consecrate morning hours to their pages. The heroic books, even if printed in
the character of our mother tongue, will always be in a language dead to
degenerate times; and we must laboriously seek the meaning of each word and
line, conjecturing a larger sense than common use permits out of what wisdom and
valor and generosity we have. The modern cheap and fertile press, with all its
translations, has done little to bring us nearer to the heroic writers of
antiquity. They seem as solitary, and the letter in which they are printed as
rare and curious, as ever. It is worth the expense of youthful days and costly
hours, if you learn only some words of an ancient language, which are raised out
of the trivialness of the street, to be perpetual suggestions and provocations.
It is not in vain that the farmer remembers and repeats the few Latin words
which he has heard. Men sometimes speak as if the study of the classics would at
length make way for more modern and practical studies; but the adventurous
student will always study classics, in whatever language they may be written and
however ancient they may be. For what are the classics but the noblest recorded
thoughts of man? They are the only oracles which are not decayed, and there are
such answers to the most modern inquiry in them as Delphi and
Dodona\footnote{Delphi and Dodona --- oracles of ancient Greece} never gave. We
might as well omit to study Nature because she is old. To read well, that is, to
read true books in a true spirit, is a noble exercise, and one that will task
the reader more than any exercise which the customs of the day esteem. It
requires a training such as the athletes underwent, the steady intention almost
of the whole life to this object. Books must be read as deliberately and
reservedly as they were written. It is not enough even to be able to speak the
language of that nation by which they are written, for there is a memorable
interval between the spoken and the written language, the language heard and the
language read. The one is commonly transitory, a sound, a tongue, a dialect
merely, almost brutish, and we learn it unconsciously, like the brutes, of our
mothers. The other is the maturity and experience of that; if that is our mother
tongue, this is our father tongue, a reserved and select expression, too
significant to be heard by the ear, which we must be born again in order to
speak. The crowds of men who merely spoke the Greek and Latin tongues in the
Middle Ages were not entitled by the accident of birth to read the works of
genius written in those languages; for these were not written in that Greek or
Latin which they knew, but in the select language of literature. They had not
learned the nobler dialects of Greece and Rome, but the very materials on which
they were written were waste paper to them, and they prized instead a cheap
contemporary literature. But when the several nations of Europe had acquired
distinct though rude written languages of their own, sufficient for the purposes
of their rising literatures, then first learning revived, and scholars were
enabled to discern from that remoteness the treasures of antiquity. What the
Roman and Grecian multitude could not hear, after the lapse of ages a few
scholars read, and a few scholars only are still reading it.

\paragraph{}
However much we may admire the orator's occasional bursts of eloquence, the
noblest written words are commonly as far behind or above the fleeting spoken
language as the firmament with its stars is behind the clouds. There are the
stars, and they who can may read them. The astronomers forever comment on and
observe them. They are not exhalations like our daily colloquies and vaporous
breath. What is called eloquence in the forum is commonly found to be rhetoric
in the study. The orator yields to the inspiration of a transient occasion, and
speaks to the mob before him, to those who can hear him; but the writer, whose
more equable life is his occasion, and who would be distracted by the event and
the crowd which inspire the orator, speaks to the intellect and health of
mankind, to all in any age who can understand him.

\paragraph{}
No wonder that Alexander\footnote{Alexander the Great of Macedon, (356 B.C.-323
    B.C.), conquered the Persian Empire; Plutarch's biography of Alexander says
    that he carried the \textit{Iliad} with him} carried the Iliad with him on
his expeditions in a precious casket. A written word is the choicest of relics.
It is something at once more intimate with us and more universal than any other
work of art. It is the work of art nearest to life itself. It may be translated
into every language, and not only be read but actually breathed from all human
lips; --- not be represented on canvas or in marble only, but be carved out of
the breath of life itself. The symbol of an ancient man's thought becomes a
modern man's speech. Two thousand summers have imparted to the monuments of
Grecian literature, as to her marbles, only a maturer golden and autumnal tint,
for they have carried their own serene and celestial atmosphere into all lands
to protect them against the corrosion of time. Books are the treasured wealth of
the world and the fit inheritance of generations and nations. Books, the oldest
and the best, stand naturally and rightfully on the shelves of every cottage.
They have no cause of their own to plead, but while they enlighten and sustain
the reader his common sense will not refuse them. Their authors are a natural
and irresistible aristocracy in every society, and, more than kings or emperors,
exert an influence on mankind. When the illiterate and perhaps scornful trader
has earned by enterprise and industry his coveted leisure and independence, and
is admitted to the circles of wealth and fashion, he turns inevitably at last to
those still higher but yet inaccessible circles of intellect and genius, and is
sensible only of the imperfection of his culture and the vanity and
insufficiency of all his riches, and further proves his good sense by the pains
which he takes to secure for his children that intellectual culture whose want
he so keenly feels; and thus it is that he becomes the founder of a family.

\paragraph{}
Those who have not learned to read the ancient classics in the language in which
they were written must have a very imperfect knowledge of the history of the
human race; for it is remarkable that no transcript of them has ever been made
into any modern tongue, unless our civilization itself may be regarded as such a
transcript. Homer has never yet been printed in English, nor \AE{}schylus, nor
Virgil\footnote{Virgil (70--19 B.C.) Roman poet (Thoreau disliked available
    translations)} even --- works as refined, as solidly done, and as beautiful
almost as the morning itself; for later writers, say what we will of their
genius, have rarely, if ever, equalled the elaborate beauty and finish and the
lifelong and heroic literary labors of the ancients. They only talk of
forgetting them who never knew them. It will be soon enough to forget them when
we have the learning and the genius which will enable us to attend to and
appreciate them. That age will be rich indeed when those relics which we call
Classics, and the still older and more than classic but even less known
Scriptures of the nations, shall have still further accumulated, when the
Vaticans shall be filled with Vedas\footnote{Brahmin religious books} and
Zendavestas\footnote{scripture of Zoroastrianism, founded by Zoroaster ca. 600
    B.C.} and Bibles, with Homers and Dantes and
Shakespeares,\footnote{References to great writers; Dante (1265--1321) was an
    Italian epic poet} and all the centuries to come shall have successively
deposited their trophies in the forum of the world. By such a pile we may hope
to scale heaven at last.

\paragraph{}
The works of the great poets have never yet been read by mankind, for only great
poets can read them. They have only been read as the multitude read the stars,
at most astrologically, not astronomically. Most men have learned to read to
serve a paltry convenience, as they have learned to cipher in order to keep
accounts and not be cheated in trade; but of reading as a noble intellectual
exercise they know little or nothing; yet this only is reading, in a high sense,
not that which lulls us as a luxury and suffers the nobler faculties to sleep
the while, but what we have to stand on tip-toe to read and devote our most
alert and wakeful hours to.

\paragraph{}
I think that having learned our letters we should read the best that is in
literature, and not be forever repeating our a-b-abs, and words of one syllable,
in the fourth or fifth classes, sitting on the lowest and foremost form all our
lives. Most men are satisfied if they read or hear read, and perchance have been
convicted by the wisdom of one good book, the Bible, and for the rest of their
lives vegetate and dissipate their faculties in what is called easy reading.
There is a work in several volumes in our Circulating Library entitled
\enquote{Little Reading,} which I thought referred to a town of that
name\footnote{The town of Reading, Mass, pronounced \enquote{redding}} which I
had not been to. There are those who, like cormorants and ostriches, can digest
all sorts of this, even after the fullest dinner of meats and vegetables, for
they suffer nothing to be wasted. If others are the machines to provide this
provender, they are the machines to read it. They read the nine thousandth tale
about Zebulon and Sophronia, and how they loved as none had ever loved before,
and neither did the course of their true love run smooth --- at any rate, how it
did run and stumble, and get up again and go on! how some poor unfortunate got
up on to a steeple, who had better never have gone up as far as the belfry; and
then, having needlessly got him up there, the happy novelist rings the bell for
all the world to come together and hear, O dear! how he did get down again! For
my part, I think that they had better metamorphose all such aspiring heroes of
universal noveldom into man weather-cocks, as they used to put heroes among the
constellations, and let them swing round there till they are rusty, and not come
down at all to bother honest men with their pranks. The next time the novelist
rings the bell I will not stir though the meeting-house burn down. \enquote{The
    Skip of the Tip-Toe-Hop, a Romance of the Middle Ages, by the celebrated
    author of \enquote*{Tittle-Tol-Tan,} to appear in monthly parts; a great
    rush; don't all come together.} All this they read with saucer eyes, and
erect and primitive curiosity, and with unwearied gizzard, whose corrugations
even yet need no sharpening, just as some little four-year-old bencher his
two-cent gilt-covered edition of Cinderella --- without any improvement, that I
can see, in the pronunciation, or accent, or emphasis, or any more skill in
extracting or inserting the moral. The result is dulness of sight, a stagnation
of the vital circulations, and a general deliquium and sloughing off of all the
intellectual faculties. This sort of gingerbread is baked daily and more
sedulously than pure wheat or rye-and-Indian in almost every oven, and finds a
surer market.

\paragraph{}
The best books are not read even by those who are called good readers. What does
our Concord culture amount to? There is in this town, with a very few
exceptions, no taste for the best or for very good books even in English
literature, whose words all can read and spell. Even the college-bred and
so-called liberally educated men here and elsewhere have really little or no
acquaintance with the English classics; and as for the recorded wisdom of
mankind, the ancient classics and Bibles, which are accessible to all who will
know of them, there are the feeblest efforts anywhere made to become acquainted
with them. I know a woodchopper, of middle age, who takes a French paper, not
for news as he says, for he is above that, but to \enquote{keep himself in
    practice,} he being a Canadian by birth; and when I ask him what he
considers the best thing he can do in this world, he says, beside this, to keep
up and add to his English. This is about as much as the college-bred generally
do or aspire to do, and they take an English paper for the purpose. One who has
just come from reading perhaps one of the best English books will find how many
with whom he can converse about it? Or suppose he comes from reading a Greek or
Latin classic in the original, whose praises are familiar even to the so-called
illiterate; he will find nobody at all to speak to, but must keep silence about
it. Indeed, there is hardly the professor in our colleges, who, if he has
mastered the difficulties of the language, has proportionally mastered the
difficulties of the wit and poetry of a Greek poet, and has any sympathy to
impart to the alert and heroic reader; and as for the sacred Scriptures, or
Bibles of mankind, who in this town can tell me even their titles? Most men do
not know that any nation but the Hebrews have had a scripture. A man, any man,
will go considerably out of his way to pick up a silver dollar; but here are
golden words, which the wisest men of antiquity have uttered, and whose worth
the wise of every succeeding age have assured us of; --- and yet we learn to
read only as far as Easy Reading, the primers and class-books, and when we leave
school, the \enquote{Little Reading,} and story-books, which are for boys and
beginners; and our reading, our conversation and thinking, are all on a very low
level, worthy only of pygmies and manikins.

\paragraph{}
I aspire to be acquainted with wiser men than this our Concord soil has
produced, whose names are hardly known here. Or shall I hear the name of
Plato\footnote{Plato (427?--347 B.C.) Greek philosopher} and never read his
book? As if Plato were my townsman and I never saw him --- my next neighbor and
I never heard him speak or attended to the wisdom of his words. But how actually
is it? His Dialogues,\footnote{In Plato's \textit{Dialoges}, characters ask each
    other questions, allowing Plato to raise various points of view and let the
    reader decide which is valid.} which contain what was immortal in him, lie
on the next shelf, and yet I never read them. We are underbred and low-lived and
illiterate; and in this respect I confess I do not make any very broad
distinction between the illiterateness of my townsman who cannot read at all and
the illiterateness of him who has learned to read only what is for children and
feeble intellects. We should be as good as the worthies of antiquity, but partly
by first knowing how good they were. We are a race of tit-men,\footnote{tit is
    an old word for \enquote{small}, in 19th cent.\ usage indicating
    \enquote{small men}} and soar but little higher in our intellectual flights
than the columns of the daily paper.

\paragraph{}
It is not all books that are as dull as their readers. There are probably words
addressed to our condition exactly, which, if we could really hear and
understand, would be more salutary than the morning or the spring to our lives,
and possibly put a new aspect on the face of things for us. How many a man has
dated a new era in his life from the reading of a book! The book exists for us,
perchance, which will explain our miracles and reveal new ones. The at present
unutterable things we may find somewhere uttered. These same questions that
disturb and puzzle and confound us have in their turn occurred to all the wise
men; not one has been omitted; and each has answered them, according to his
ability, by his words and his life. Moreover, with wisdom we shall learn
liberality. The solitary hired man on a farm in the outskirts of Concord, who
has had his second birth and peculiar religious experience, and is driven as he
believes into the silent gravity and exclusiveness by his faith, may think it is
not true; but Zoroaster,\footnote{Persian prophet ca. 600 B.C., also known as
    Zarathustra} thousands of years ago, travelled the same road and had the
same experience; but he, being wise, knew it to be universal, and treated his
neighbors accordingly, and is even said to have invented and established worship
among men. Let him humbly commune with Zoroaster then, and through the
liberalizing influence of all the worthies, with Jesus Christ himself, and let
\enquote{our church} go by the board.

\paragraph{}
We boast that we belong to the Nineteenth Century and are making the most rapid
strides of any nation. But consider how little this village does for its own
culture. I do not wish to flatter my townsmen, nor to be flattered by them, for
that will not advance either of us. We need to be provoked --- goaded like oxen,
as we are, into a trot. We have a comparatively decent system of common schools,
schools for infants only; but excepting the half-starved
Lyceum\footnote{organization that sponsers cultural events} in the winter, and
latterly the puny beginning of a library suggested by the State, no school for
ourselves. We spend more on almost any article of bodily aliment or ailment than
on our mental aliment. It is time that we had uncommon schools, that we did not
leave off our education when we begin to be men and women. It is time that
villages were universities, and their elder inhabitants the fellows of
universities, with leisure --- if they are, indeed, so well off --- to pursue
liberal studies the rest of their lives. Shall the world be confined to one
Paris or one Oxford forever? Cannot students be boarded here and get a liberal
education under the skies of Concord? Can we not hire some
Abelard\footnote{Peter Abelard (1079--1142) French philosopher and theologian}
to lecture to us? Alas! what with foddering the cattle and tending the store, we
are kept from school too long, and our education is sadly neglected. In this
country, the village should in some respects take the place of the nobleman of
Europe. It should be the patron of the fine arts. It is rich enough. It wants
only the magnanimity and refinement. It can spend money enough on such things as
farmers and traders value, but it is thought Utopian\footnote{relating to an
    imagined ideal place} to propose spending money for things which more
intelligent men know to be of far more worth. This town has spent seventeen
thousand dollars on a town-house, thank fortune or politics, but probably it
will not spend so much on living wit, the true meat to put into that shell, in a
hundred years. The one hundred and twenty-five dollars annually subscribed for a
Lyceum in the winter is better spent than any other equal sum raised in the
town. If we live in the Nineteenth Century, why should we not enjoy the
advantages which the Nineteenth Century offers? Why should our life be in any
respect provincial? If we will read newspapers, why not skip the gossip of
Boston and take the best newspaper in the world at once? --- not be sucking the
pap of \enquote{neutral family} papers, or browsing \enquote{Olive
    Branches}\footnote{publications that do not have editorial opinions} here in
New England. Let the reports of all the learned societies come to us, and we
will see if they know anything. Why should we leave it to Harper \& Brothers and
Redding \& Co.\footnote{publishers in New York and Boston} to select our
reading? As the nobleman of cultivated taste surrounds himself with whatever
conduces to his culture --- genius --- learning --- wit --- books --- paintings
--- statuary --- music --- philosophical instruments, and the like; so let the
village do --- not stop short at a pedagogue, a parson, a sexton, a parish
library, and three selectmen, because our Pilgrim forefathers got through a cold
winter once on a bleak rock with these. To act collectively is according to the
spirit of our institutions; and I am confident that, as our circumstances are
more flourishing, our means are greater than the nobleman's. New England can
hire all the wise men in the world to come and teach her, and board them round
the while, and not be provincial at all. That is the uncommon school we want.
Instead of noblemen, let us have noble villages of men. If it is necessary, omit
one bridge over the river, go round a little there, and throw one arch at least
over the darker gulf of ignorance which surrounds us.

\end{document}

