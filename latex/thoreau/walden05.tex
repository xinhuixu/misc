% Adapted from http://thoreau.eserver.org/walden05.html

\documentclass[letterpaper,12pt]{article}
\usepackage[margin=0.964in]{geometry}
\usepackage{csquotes}
\usepackage{epigraph}
\usepackage{scrextend}
\usepackage{xspace}

\title{\vspace{-2em}Walden: Solitude}
\author{Henry David Thoreau}
\date{1849}

\renewcommand{\theparagraph}{[\arabic{paragraph}]}
\setcounter{secnumdepth}{4}

\deffootnote[1em]{1.5em}{1em}{\textsuperscript{\thefootnotemark}}

\setlength{\epigraphwidth}{\textwidth}
\setlength\epigraphrule{0pt}
\renewcommand{\textflush}{flushepinormal}

\newcommand{\el}{{\,\ldots}\xspace}

\begin{document}
\maketitle

\hrule
\epigraph{\enquote{When you get into a railway car you want a continent, the man
        in his carriage requires a township; but a walker like Thoreau finds as
        much and more along the shores of Walden Pond.}}{---John Burroughs,
    \textit{The Galaxy}, June 1873}
\vspace{-1em}
\epigraph{The reference below to unexpected visitors shows that Thoreau was
    reasonably close to town, with friends that might walk out to visit, and not
    the hermit that some readers have assumed.}
\hrule

\paragraph{}
This is a delicious evening, when the whole body is one sense, and imbibes
delight through every pore. I go and come with a strange liberty in Nature, a
part of herself. As I walk along the stony shore of the pond in my
shirt-sleeves, though it is cool as well as cloudy and windy, and I see nothing
special to attract me, all the elements are unusually congenial to me. The
bullfrogs trump to usher in the night, and the note of the whip-poor-will is
borne on the rippling wind from over the water. Sympathy with the fluttering
alder and poplar leaves almost takes away my breath; yet, like the lake, my
serenity is rippled but not ruffled. These small waves raised by the evening
wind are as remote from storm as the smooth reflecting surface. Though it is now
dark, the wind still blows and roars in the wood, the waves still dash, and some
creatures lull the rest with their notes. The repose is never complete. The
wildest animals do not repose, but seek their prey now; the fox, and skunk, and
rabbit, now roam the fields and woods without fear. They are Nature's watchmen
--- links which connect the days of animated life.

\paragraph{}
When I return to my house I find that visitors have been there and left their
cards, either a bunch of flowers, or a wreath of evergreen, or a name in pencil
on a yellow walnut leaf or a chip. They who come rarely to the woods take some
little piece of the forest into their hands to play with by the way, which they
leave, either intentionally or accidentally. One has peeled a willow wand, woven
it into a ring, and dropped it on my table. I could always tell if visitors had
called in my absence, either by the bended twigs or grass, or the print of their
shoes, and generally of what sex or age or quality they were by some slight
trace left, as a flower dropped, or a bunch of grass plucked and thrown away,
even as far off as the railroad, half a mile distant, or by the lingering odor
of a cigar or pipe. Nay, I was frequently notified of the passage of a traveller
along the highway sixty rods off by the scent of his pipe.

\paragraph{}
There is commonly sufficient space about us. Our horizon is never quite at our
elbows. The thick wood is not just at our door, nor the pond, but somewhat is
always clearing, familiar and worn by us, appropriated and fenced in some way,
and reclaimed from Nature. For what reason have I this vast range and circuit,
some square miles of unfrequented forest, for my privacy, abandoned to me by
men? My nearest neighbor is a mile distant, and no house is visible from any
place but the hill-tops within half a mile of my own. I have my horizon bounded
by woods all to myself; a distant view of the railroad where it touches the pond
on the one hand, and of the fence which skirts the woodland road on the other.
But for the most part it is as solitary where I live as on the prairies. It is
as much Asia or Africa as New England. I have, as it were, my own sun and moon
and stars, and a little world all to myself. At night there was never a
traveller passed my house, or knocked at my door, more than if I were the first
or last man; unless it were in the spring, when at long intervals some came from
the village to fish for pouts --- they plainly fished much more in the Walden
Pond of their own natures, and baited their hooks with darkness --- but they
soon retreated, usually with light baskets, and left \enquote{the world to
    darkness and to me,} and the black kernel of the night was never profaned by
any human neighborhood. I believe that men are generally still a little afraid
of the dark, though the witches are all hung, and Christianity and candles have
been introduced.

\paragraph{}
Yet I experienced sometimes that the most sweet and tender, the most innocent
and encouraging society may be found in any natural object, even for the poor
misanthrope and most melancholy man. There can be no very black melancholy to
him who lives in the midst of Nature and has his senses still. There was never
yet such a storm but it was \AE{}olian music\footnote{in Greek mythology, the
    Aeolian harp was the instrument of \AE{}olus, god of wind. The ancient
    Greeks made Aeolian harps that were played by moving air} to a healthy and
innocent ear.  Nothing can rightly compel a simple and brave man to a vulgar
sadness. While I enjoy the friendship of the seasons I trust that nothing can
make life a burden to me. The gentle rain which waters my beans and keeps me in
the house today is not drear and melancholy, but good for me too. Though it
prevents my hoeing them, it is of far more worth than my hoeing. If it should
continue so long as to cause the seeds to rot in the ground and destroy the
potatoes in the low lands, it would still be good for the grass on the uplands,
and, being good for the grass, it would be good for me. Sometimes, when I
compare myself with other men, it seems as if I were more favored by the gods
than they, beyond any deserts that I am conscious of; as if I had a warrant and
surety at their hands which my fellows have not, and were especially guided and
guarded. I do not flatter myself, but if it be possible they flatter me. I have
never felt lonesome, or in the least oppressed by a sense of solitude, but once,
and that was a few weeks after I came to the woods, when, for an hour, I doubted
if the near neighborhood of man was not essential to a serene and healthy life.
To be alone was something unpleasant. But I was at the same time conscious of a
slight insanity in my mood, and seemed to foresee my recovery. In the midst of a
gentle rain while these thoughts prevailed, I was suddenly sensible of such
sweet and beneficent society in Nature, in the very pattering of the drops, and
in every sound and sight around my house, an infinite and unaccountable
friendliness all at once like an atmosphere sustaining me, as made the fancied
advantages of human neighborhood insignificant, and I have never thought of them
since. Every little pine needle expanded and swelled with sympathy and
befriended me. I was so distinctly made aware of the presence of something
kindred to me, even in scenes which we are accustomed to call wild and dreary,
and also that the nearest of blood to me and humanest was not a person nor a
villager, that I thought no place could ever be strange to me again.

\begin{verse}
    \enquote{Mourning untimely consumes the sad; \\
        Few are their days in the land of the living, \\
        Beautiful daughter of Toscar.}\footnote{James Macpherson (1736--1796)
        from \textit{Croma}, poetry of \enquote{Ossian}, supposed 3rd cent.\
        Gaelic poet, later established as a forgery by Macpherson}
\end{verse}

\paragraph{}
Some of my pleasantest hours were during the long rain-storms in the spring or
fall, which confined me to the house for the afternoon as well as the forenoon,
soothed by their ceaseless roar and pelting; when an early twilight ushered in a
long evening in which many thoughts had time to take root and unfold themselves.
In those driving northeast rains which tried the village houses so, when the
maids stood ready with mop and pail in front entries to keep the deluge out, I
sat behind my door in my little house, which was all entry, and thoroughly
enjoyed its protection. In one heavy thunder-shower the lightning struck a large
pitch pine across the pond, making a very conspicuous and perfectly regular
spiral groove from top to bottom, an inch or more deep, and four or five inches
wide, as you would groove a walking-stick. I passed it again the other day, and
was struck with awe on looking up and beholding that mark, now more distinct
than ever, where a terrific and resistless bolt came down out of the harmless
sky eight years ago.\footnote{Thoreau lived at Walden from 1845 to 1847.
    \textit{Walden} was not published until 1854} Men frequently say to me,
\enquote{I should think you would feel lonesome down there, and want to be
    nearer to folks, rainy and snowy days and nights especially.} I am tempted
to reply to such --- This whole earth which we inhabit is but a point in space.
How far apart, think you, dwell the two most distant inhabitants of yonder star,
the breadth of whose disk cannot be appreciated by our instruments? Why should I
feel lonely? Is not our planet in the Milky Way? This which you put seems to me
not to be the most important question. What sort of space is that which
separates a man from his fellows and makes him solitary? I have found that no
exertion of the legs can bring two minds much nearer to one another. What do we
want most to dwell near to? Not to many men surely, the depot, the post-office,
the bar-room, the meeting-house, the school-house, the grocery, Beacon
Hill,\footnote{fashionable section of Boston} or the Five
Points,\footnote{former disreputable section of New York City, between the
    current NY City Hall and Chinatown} where men most congregate, but to the
perennial source of our life, whence in all our experience we have found that to
issue, as the willow stands near the water and sends out its roots in that
direction. This will vary with different natures, but this is the place where a
wise man will dig his cellar\el{}. I one evening overtook one of my townsmen,
who has accumulated what is called \enquote{a handsome property} --- though I
never got a fair view of it --- on the Walden road, driving a pair of cattle to
market, who inquired of me how I could bring my mind to give up so many of the
comforts of life. I answered that I was very sure I liked it passably well; I
was not joking. And so I went home to my bed, and left him to pick his way
through the darkness and the mud to Brighton --- or Bright-town --- which place
he would reach some time in the morning.

\paragraph{}
Any prospect of awakening or coming to life to a dead man makes indifferent all
times and places. The place where that may occur is always the same, and
indescribably pleasant to all our senses. For the most part we allow only
outlying and transient circumstances to make our occasions. They are, in fact,
the cause of our distraction. Nearest to all things is that power which fashions
their being. Next to us the grandest laws are continually being executed. Next
to us is not the workman whom we have hired, with whom we love so well to talk,
but the workman whose work we are.

\paragraph{}
\enquote{How vast and profound is the influence of the subtile powers of Heaven
    and of Earth!}

\paragraph{}
\enquote{We seek to perceive them, and we do not see them; we seek to hear them,
    and we do not hear them; identified with the substance of things, they
    cannot be separated from them.}

\paragraph{}
\enquote{They cause that in all the universe men purify and sanctify their
    hearts, and clothe themselves in their holiday garments to offer sacrifices
    and oblations to their ancestors. It is an ocean of subtile intelligences.
    They are everywhere, above us, on our left, on our right; they environ us on
    all sides.}\footnote{Confucius (1551--1479 B.C.) Chinese philosopher, three
    paragraphs in quotes are from \textit{Doctrine of the Mean}}

\paragraph{}
We are the subjects of an experiment which is not a little interesting to me.
Can we not do without the society of our gossips a little while under these
circumstances --- have our own thoughts to cheer us? Confucius says truly,
\enquote{Virtue does not remain as an abandoned orphan; it must of necessity
    have neighbors.}\footnote{\textit{Confucian Analects}}

\paragraph{}
With thinking we may be beside ourselves in a sane sense. By a conscious effort
of the mind we can stand aloof from actions and their consequences; and all
things, good and bad, go by us like a torrent. We are not wholly involved in
Nature. I may be either the driftwood in the stream, or Indra\footnote{in
    Hinduism, chief of the Vedic gods, god of thunder, \& rain} in the sky
looking down on it. I may be affected by a theatrical exhibition; on the other
hand, I may not be affected by an actual event which appears to concern me much
more. I only know myself as a human entity; the scene, so to speak, of thoughts
and affections; and am sensible of a certain doubleness by which I can stand as
remote from myself as from another. However intense my experience, I am
conscious of the presence and criticism of a part of me, which, as it were, is
not a part of me, but spectator, sharing no experience, but taking note of it,
and that is no more I than it is you. When the play, it may be the tragedy, of
life is over, the spectator goes his way. It was a kind of fiction, a work of
the imagination only, so far as he was concerned. This doubleness may easily
make us poor neighbors and friends sometimes.

\paragraph{}
I find it wholesome to be alone the greater part of the time. To be in company,
even with the best, is soon wearisome and dissipating. I love to be alone. I
never found the companion that was so companionable as solitude. We are for the
most part more lonely when we go abroad among men than when we stay in our
chambers. A man thinking or working is always alone, let him be where he will.
Solitude is not measured by the miles of space that intervene between a man and
his fellows. The really diligent student in one of the crowded hives of
Cambridge College is as solitary as a dervish in the desert. The farmer can work
alone in the field or the woods all day, hoeing or chopping, and not feel
lonesome, because he is employed; but when he comes home at night he cannot sit
down in a room alone, at the mercy of his thoughts, but must be where he can
\enquote{see the folks,} and recreate, and as he thinks remunerate himself for
his day's solitude; and hence he wonders how the student can sit alone in the
house all night and most of the day without ennui and \enquote{the blues}; but
he does not realize that the student, though in the house, is still at work in
his field, and chopping in his woods, as the farmer in his, and in turn seeks
the same recreation and society that the latter does, though it may be a more
condensed form of it.

\paragraph{}
Society is commonly too cheap. We meet at very short intervals, not having had
time to acquire any new value for each other. We meet at meals three times a
day, and give each other a new taste of that old musty cheese that we are. We
have had to agree on a certain set of rules, called etiquette and politeness, to
make this frequent meeting tolerable and that we need not come to open war. We
meet at the post-office, and at the sociable, and about the fireside every
night; we live thick and are in each other's way, and stumble over one another,
and I think that we thus lose some respect for one another. Certainly less
frequency would suffice for all important and hearty communications. Consider
the girls in a factory --- never alone, hardly in their dreams. It would be
better if there were but one inhabitant to a square mile, as where I live. The
value of a man is not in his skin, that we should touch him.

\paragraph{}
I have heard of a man lost in the woods and dying of famine and exhaustion at
the foot of a tree, whose loneliness was relieved by the grotesque visions with
which, owing to bodily weakness, his diseased imagination surrounded him, and
which he believed to be real. So also, owing to bodily and mental health and
strength, we may be continually cheered by a like but more normal and natural
society, and come to know that we are never alone.

\paragraph{}
I have a great deal of company in my house; especially in the morning, when
nobody calls. Let me suggest a few comparisons, that some one may convey an idea
of my situation. I am no more lonely than the loon in the pond that laughs so
loud, or than Walden Pond itself. What company has that lonely lake, I pray? And
yet it has not the blue devils,\footnote{hypochondriac melancholy} but the blue
angels in it, in the azure tint of its waters. The sun is alone, except in thick
weather, when there sometimes appear to be two, but one is a mock sun. God is
alone --- but the devil, he is far from being alone; he sees a great deal of
company; he is legion. I am no more lonely than a single mullein or dandelion in
a pasture, or a bean leaf, or sorrel, or a horse-fly, or a bumblebee. I am no
more lonely than the Mill Brook, or a weathercock, or the north star, or the
south wind, or an April shower, or a January thaw, or the first spider in a new
house.

\paragraph{}
I have occasional visits in the long winter evenings, when the snow falls fast
and the wind howls in the wood, from an old settler and original proprietor, who
is reported to have dug Walden Pond, and stoned it, and fringed it with pine
woods; who tells me stories of old time and of new eternity; and between us we
manage to pass a cheerful evening with social mirth and pleasant views of
things, even without apples or cider --- a most wise and humorous friend, whom I
love much, who keeps himself more secret than ever did Goffe or
Whalley;\footnote{William Goffe, Edward Whalley, indicted for killing Charles I
    of England, lived in hiding in America} and though he is thought to be dead,
none can show where he is buried. An elderly dame, too, dwells in my
neighborhood, invisible to most persons, in whose odorous herb garden I love to
stroll sometimes, gathering simples and listening to her fables; for she has a
genius of unequalled fertility, and her memory runs back farther than mythology,
and she can tell me the original of every fable, and on what fact every one is
founded, for the incidents occurred when she was young. A ruddy and lusty old
dame, who delights in all weathers and seasons, and is likely to outlive all her
children yet.

\paragraph{}
The indescribable innocence and beneficence of Nature --- of sun and wind and
rain, of summer and winter --- such health, such cheer, they afford forever! and
such sympathy have they ever with our race, that all Nature would be affected,
and the sun's brightness fade, and the winds would sigh humanely, and the clouds
rain tears, and the woods shed their leaves and put on mourning in midsummer, if
any man should ever for a just cause grieve. Shall I not have intelligence with
the earth? Am I not partly leaves and vegetable mould myself?

\paragraph{}
What is the pill which will keep us well, serene, contented? Not my or thy
great-grandfather's, but our great-grandmother Nature's universal, vegetable,
botanic medicines, by which she has kept herself young always, outlived so many
old Parrs\footnote{Thomas Parr was an Englishman said to have lived 152 years}
in her day, and fed her health with their decaying fatness. For my panacea,
instead of one of those quack vials of a mixture dipped from Acheron\footnote{in
    Greek mythology, a river in Hades} and the Dead Sea,\footnote{large salt
    lake bordering Israel \& Jordan} which come out of those long shallow
black-schooner looking wagons which we sometimes see made to carry bottles, let
me have a draught of undiluted morning air. Morning air! If men will not drink
of this at the fountainhead of the day, why, then, we must even bottle up some
and sell it in the shops, for the benefit of those who have lost their
subscription ticket to morning time in this world. But remember, it will not
keep quite till noonday even in the coolest cellar, but drive out the stopples
long ere that and follow westward the steps of Aurora.\footnote{in Roman
    mythology, goddess of the dawn} I am no worshipper of Hygeia,\footnote{in
    Greek mythology, goddess of health} who was the daughter of that old
herb-doctor \AE{}sculapius,\footnote{in Greek mythology, god of medicine, father
    of Hygeia} and who is represented on monuments holding a serpent in one
hand, and in the other a cup out of which the serpent sometimes drinks; but
rather of Hebe,\footnote{in Greek mythology, goddess of youth} cup-bearer to
Jupiter,\footnote{in Roman mythology, chief of the gods} who was the daughter of
Juno and wild lettuce,\footnote{in Roman mythology, queen of heaven, conceived
    Hebe after eating lettuce} and who had the power of restoring gods and men
to the vigor of youth. She was probably the only thoroughly sound-conditioned,
healthy, and robust young lady that ever walked the globe, and wherever she came
it was spring.

\end{document}

