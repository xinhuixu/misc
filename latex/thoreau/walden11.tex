% Adapted from http://thoreau.eserver.org/walden11.html

\documentclass[letterpaper,12pt]{article}
\usepackage[margin=0.75in]{geometry}
\usepackage{csquotes}
\usepackage{epigraph}
\usepackage{scrextend}
\usepackage{xspace}

\title{\vspace{-2em}Walden: Higher Laws}
\author{Henry David Thoreau}
\date{1849}

\renewcommand{\theparagraph}{[\arabic{paragraph}]}
\setcounter{secnumdepth}{4}

\deffootnote[1em]{1.5em}{1em}{\textsuperscript{\thefootnotemark}}

\setlength{\epigraphwidth}{\textwidth}
\setlength\epigraphrule{0pt}
\renewcommand{\textflush}{flushepinormal}

\newcommand{\el}{{\,\ldots}\xspace}

\begin{document}
\maketitle

\paragraph{} As I came home through the woods with my string of fish, trailing
my pole, it being now quite dark, I caught a glimpse of a woodchuck stealing
across my path, and felt a strange thrill of savage delight, and was strongly
tempted to seize and devour him raw; not that I was hungry then, except for that
wildness which he represented. Once or twice, however, while I lived at the
pond, I found myself ranging the woods, like a half-starved hound, with a
strange abandonment, seeking some kind of venison which I might devour, and no
morsel could have been too savage for me. The wildest scenes had become
unaccountably familiar. I found in myself, and still find, an instinct toward a
higher, or, as it is named, spiritual life, as do most men, and another toward a
primitive rank and savage one, and I reverence them both. I love the wild not
less than the good. The wildness and adventure that are in fishing still
recommended it to me. I like sometimes to take rank hold on life and spend my
day more as the animals do.  Perhaps I have owed to this employment and to
hunting, when quite young, my closest acquaintance with Nature. They early
introduce us to and detain us in scenery with which otherwise, at that age, we
should have little acquaintance.  Fishermen, hunters, woodchoppers, and others,
spending their lives in the fields and woods, in a peculiar sense a part of
Nature themselves, are often in a more favorable mood for observing her, in the
intervals of their pursuits, than philosophers or poets even, who approach her
with expectation. She is not afraid to exhibit herself to them. The traveler on
the prairie is naturally a hunter, on the head waters of the Missouri and
Columbia a trapper, and at the Falls of St.\ Mary a fisherman. He who is only a
traveler learns things at second-hand and by the halves, and is poor authority.
We are most interested when science reports what those men already know
practically or instinctively, for that alone is a true humanity, or account of
human experience.

\paragraph{} They mistake who assert that the Yankee has few amusements, because
he has not so many public holidays, and men and boys do not play so many games
as they do in England, for here the more primitive but solitary amusements of
hunting, fishing, and the like have not yet given place to the former. Almost
every New England boy among my contemporaries shouldered a fowling-piece between
the ages of ten and fourteen; and his hunting and fishing grounds were not
limited, like the preserves of an English nobleman, but were more boundless even
than those of a savage. No wonder, then, that he did not oftener stay to play on
the common.  But already a change is taking place, owing, not to an increased
humanity, but to an increased scarcity of game, for perhaps the hunter is the
greatest friend of the animals hunted, not excepting the Humane Society.

\paragraph{} Moreover, when at the pond, I wished sometimes to add fish to my
fare for variety. I have actually fished from the same kind of necessity that
the first fishers did. Whatever humanity I might conjure up against it was all
factitious, and concerned my philosophy more than my feelings. I speak of
fishing only now, for I had long felt differently about fowling, and sold my gun
before I went to the woods. Not that I am less humane than others, but I did not
perceive that my feelings were much affected. I did not pity the fishes nor the
worms. This was habit. As for fowling, during the last years that I carried a
gun my excuse was that I was studying ornithology, and sought only new or rare
birds. But I confess that I am now inclined to think that there is a finer way
of studying ornithology than this. It requires so much closer attention to the
habits of the birds, that, if for that reason only, I have been willing to omit
the gun. Yet notwithstanding the objection on the score of humanity, I am
compelled to doubt if equally valuable sports are ever substituted for these;
and when some of my friends have asked me anxiously about their boys, whether
they should let them hunt, I have answered, yes --- remembering that it was one
of the best parts of my education --- make them hunters, though sportsmen only
at first, if possible, mighty hunters at last, so that they shall not find game
large enough for them in this or any vegetable wilderness --- hunters as well as
fishers of men. Thus far I am of the opinion of Chaucer's nun, who

\begin{verse}
    \enquote{yave not of the text a pulled hen \\
        That saith that hunters ben not holy men.}\footnote{Geoffrey Chaucer (1340?--1400) \textit{Canterberry Tales}}
\end{verse}

There is a period in the history of the individual, as of the race, when the
hunters are the \enquote{best men,} as the Algonquins\footnote{American Indian
    tribe, originally north of the St.\ Lawrence River} called them. We cannot
but pity the boy who has never fired a gun; he is no more humane, while his
education has been sadly neglected. This was my answer with respect to those
youths who were bent on this pursuit, trusting that they would soon outgrow it.
No humane being, past the thoughtless age of boyhood, will wantonly murder any
creature which holds its life by the same tenure that he does. The hare in its
extremity cries like a child. I warn you, mothers, that my sympathies do not
always make the usual philanthropic distinctions.

\paragraph{} Such is oftenest the young man's introduction to the forest, and
the most original part of himself. He goes thither at first as a hunter and
fisher, until at last, if he has the seeds of a better life in him, he
distinguishes his proper objects, as a poet or naturalist it may be, and leaves
the gun and fish-pole behind. The mass of men are still and always young in this
respect. In some countries a hunting parson is no uncommon sight. Such a one
might make a good shepherd's dog, but is far from being the Good Shepherd. I
have been surprised to consider that the only obvious employment, except
wood-chopping, ice-cutting, or the like business, which ever to my knowledge
detained at Walden Pond for a whole half-day any of my fellow-citizens, whether
fathers or children of the town, with just one exception, was fishing. Commonly
they did not think that they were lucky, or well paid for their time, unless
they got a long string of fish, though they had the opportunity of seeing the
pond all the while. They might go there a thousand times before the sediment of
fishing would sink to the bottom and leave their purpose pure; but no doubt such
a clarifying process would be going on all the while. The Governor and his
Council faintly remember the pond, for they went a-fishing there when they were
boys; but now they are too old and dignified to go a-fishing, and so they know
it no more forever. Yet even they expect to go to heaven at last. If the
legislature regards it, it is chiefly to regulate the number of hooks to be used
there; but they know nothing about the hook of hooks with which to angle for the
pond itself, impaling the legislature for a bait. Thus, even in civilized
communities, the embryo man passes through the hunter stage of development.

\paragraph{} I have found repeatedly, of late years, that I cannot fish without
falling a little in self-respect. I have tried it again and again. I have skill
at it, and, like many of my fellows, a certain instinct for it, which revives
from time to time, but always when I have done I feel that it would have been
better if I had not fished. I think that I do not mistake. It is a faint
intimation, yet so are the first streaks of morning. There is unquestionably
this instinct in me which belongs to the lower orders of creation; yet with
every year I am less a fisherman, though without more humanity or even wisdom;
at present I am no fisherman at all. But I see that if I were to live in a
wilderness I should again be tempted to become a fisher and hunter in earnest.
Beside, there is something essentially unclean about this diet and all flesh,
and I began to see where housework commences, and whence the endeavor, which
costs so much, to wear a tidy and respectable appearance each day, to keep the
house sweet and free from all ill odors and sights. Having been my own butcher
and scullion and cook, as well as the gentleman for whom the dishes were served
up, I can speak from an unusually complete experience. The practical objection
to animal food in my case was its uncleanness; and besides, when I had caught
and cleaned and cooked and eaten my fish, they seemed not to have fed me
essentially. It was insignificant and unnecessary, and cost more than it came
to. A little bread or a few potatoes would have done as well, with less trouble
and filth. Like many of my contemporaries, I had rarely for many years used
animal food, or tea, or coffee, etc.; not so much because of any ill effects
which I had traced to them, as because they were not agreeable to my
imagination. The repugnance to animal food is not the effect of experience, but
is an instinct. It appeared more beautiful to live low and fare hard in many
respects; and though I never did so, I went far enough to please my imagination.
I believe that every man who has ever been earnest to preserve his higher or
poetic faculties in the best condition has been particularly inclined to abstain
from animal food, and from much food of any kind. It is a significant fact,
stated by entomologists --- I find it in Kirby and Spence\footnote{William Kirby
    (1759--1850), William Spence (1783--1860), British entomologists, wrote
    \textit{An Introduction to Entomology}} --- that \enquote{some insects in
    their perfect state, though furnished with organs of feeding, make no use of
    them}; and they lay it down as \enquote{a general rule, that almost all
    insects in this state eat much less than in that of larvae. The voracious
    caterpillar when transformed into a butterfly\dots and the gluttonous maggot
    when become a fly} content themselves with a drop or two of honey or some
other sweet liquid. The abdomen under the wings of the butterfly still
represents the larva. This is the tidbit which tempts his insectivorous fate.
The gross feeder is a man in the larva state; and there are whole nations in
that condition, nations without fancy or imagination, whose vast abdomens betray
them.

\paragraph{} It is hard to provide and cook so simple and clean a diet as will
not offend the imagination; but this, I think, is to be fed when we feed the
body; they should both sit down at the same table. Yet perhaps this may be done.
The fruits eaten temperately need not make us ashamed of our appetites, nor
interrupt the worthiest pursuits. But put an extra condiment into your dish, and
it will poison you. It is not worth the while to live by rich cookery. Most men
would feel shame if caught preparing with their own hands precisely such a
dinner, whether of animal or vegetable food, as is every day prepared for them
by others. Yet till this is otherwise we are not civilized, and, if gentlemen
and ladies, are not true men and women. This certainly suggests what change is
to be made. It may be vain to ask why the imagination will not be reconciled to
flesh and fat. I am satisfied that it is not. Is it not a reproach that man is a
carnivorous animal? True, he can and does live, in a great measure, by preying
on other animals; but this is a miserable way --- as any one who will go to
snaring rabbits, or slaughtering lambs, may learn --- and he will be regarded as
a benefactor of his race who shall teach man to confine himself to a more
innocent and wholesome diet. Whatever my own practice may be, I have no doubt
that it is a part of the destiny of the human race, in its gradual improvement,
to leave off eating animals, as surely as the savage tribes have left off eating
each other when they came in contact with the more civilized.

\paragraph{} If one listens to the faintest but constant suggestions of his
genius, which are certainly true, he sees not to what extremes, or even
insanity, it may lead him; and yet that way, as he grows more resolute and
faithful, his road lies. The faintest assured objection which one healthy man
feels will at length prevail over the arguments and customs of mankind. No man
ever followed his genius till it misled him. Though the result were bodily
weakness, yet perhaps no one can say that the consequences were to be regretted,
for these were a life in conformity to higher principles. If the day and the
night are such that you greet them with joy, and life emits a fragrance like
flowers and sweet-scented herbs, is more elastic, more starry, more immortal ---
that is your success. All nature is your congratulation, and you have cause
momentarily to bless yourself. The greatest gains and values are farthest from
being appreciated. We easily come to doubt if they exist. We soon forget them.
They are the highest reality. Perhaps the facts most astounding and most real
are never communicated by man to man. The true harvest of my daily life is
somewhat as intangible and indescribable as the tints of morning or evening. It
is a little star-dust caught, a segment of the rainbow which I have clutched.

\paragraph{} Yet, for my part, I was never unusually squeamish; I could
sometimes eat a fried rat with a good relish, if it were necessary. I am glad to
have drunk water so long, for the same reason that I prefer the natural sky to
an opium-eater's heaven. I would fain keep sober always; and there are infinite
degrees of drunkenness. I believe that water is the only drink for a wise man;
wine is not so noble a liquor; and think of dashing the hopes of a morning with
a cup of warm coffee, or of an evening with a dish of tea! Ah, how low I fall
when I am tempted by them! Even music may be intoxicating. Such apparently
slight causes destroyed Greece and Rome, and will destroy England and America.
Of all ebriosity, who does not prefer to be intoxicated by the air he breathes?
I have found it to be the most serious objection to coarse labors long
continued, that they compelled me to eat and drink coarsely also. But to tell
the truth, I find myself at present somewhat less particular in these respects.
I carry less religion to the table, ask no blessing; not because I am wiser than
I was, but, I am obliged to confess, because, however much it is to be
regretted, with years I have grown more coarse and indifferent. Perhaps these
questions are entertained only in youth, as most believe of poetry. My practice
is \enquote{nowhere,} my opinion is here. Nevertheless I am far from regarding
myself as one of those privileged ones to whom the Ved refers when it says, that
\enquote{he who has true faith in the Omnipresent Supreme Being may eat all that
    exists,} that is, is not bound to inquire what is his food, or who prepares
it; and even in their case it is to be observed, as a Hindoo
commentator\footnote{Raja Rammohun Roy (1772--1833) from a translation of Hindu
    scripture} has remarked, that the Vedant limits this privilege to
\enquote{the time of distress.}

\paragraph{} Who has not sometimes derived an inexpressible satisfaction from
his food in which appetite had no share? I have been thrilled to think that I
owed a mental perception to the commonly gross sense of taste, that I have been
inspired through the palate, that some berries which I had eaten on a hillside
had fed my genius. \enquote{The soul not being mistress of herself,} says
Thseng-tseu,\footnote{Confucius (551?--478? B.C.) Chinese philosopher \&
    teacher} \enquote{one looks, and one does not see; one listens, and one does
    not hear; one eats, and one does not know the savor of food.} He who
distinguishes the true savor of his food can never be a glutton; he who does not
cannot be otherwise. A puritan may go to his brown-bread crust with as gross an
appetite as ever an alderman to his turtle. Not that food which entereth into
the mouth defileth a man, but the appetite with which it is eaten. It is neither
the quality nor the quantity, but the devotion to sensual savors; when that
which is eaten is not a viand to sustain our animal, or inspire our spiritual
life, but food for the worms that possess us. If the hunter has a taste for
mud-turtles, muskrats, and other such savage tidbits, the fine lady indulges a
taste for jelly made of a calf's foot, or for sardines from over the sea, and
they are even. He goes to the mill-pond, she to her preserve-pot. The wonder is
how they, how you and I, can live this slimy, beastly life, eating and drinking.

\paragraph{} Our whole life is startlingly moral. There is never an instant's
truce between virtue and vice. Goodness is the only investment that never fails.
In the music of the harp which trembles round the world it is the insisting on
this which thrills us. The harp is the traveling patterer for the Universe's
Insurance Company, recommending its laws, and our little goodness is all the
assessment that we pay. Though the youth at last grows indifferent, the laws of
the universe are not indifferent, but are forever on the side of the most
sensitive. Listen to every zephyr for some reproof, for it is surely there, and
he is unfortunate who does not hear it. We cannot touch a string or move a stop
but the charming moral transfixes us. Many an irksome noise, go a long way off,
is heard as music, a proud, sweet satire on the meanness of our lives.

\paragraph{} We are conscious of an animal in us, which awakens in proportion as
our higher nature slumbers. It is reptile and sensual, and perhaps cannot be
wholly expelled; like the worms which, even in life and health, occupy our
bodies. Possibly we may withdraw from it, but never change its nature. I fear
that it may enjoy a certain health of its own; that we may be well, yet not
pure. The other day I picked up the lower jaw of a hog, with white and sound
teeth and tusks, which suggested that there was an animal health and vigor
distinct from the spiritual. This creature succeeded by other means than
temperance and purity. \enquote{That in which men differ from brute beasts,}
says Mencius,\footnote{Meng-tse (372?--287? B.C.) Chinese philosopher, follower
    of Confucius} \enquote{is a thing very inconsiderable; the common herd lose
    it very soon; superior men preserve it carefully.} Who knows what sort of
life would result if we had attained to purity? If I knew so wise a man as could
teach me purity I would go to seek him forthwith. \enquote{A command over our
    passions, and over the external senses of the body, and good acts, are
    declared by the Ved to be indispensable in the mind's approximation to God.}
Yet the spirit can for the time pervade and control every member and function of
the body, and transmute what in form is the grossest sensuality into purity and
devotion. The generative energy, which, when we are loose, dissipates and makes
us unclean, when we are continent invigorates and inspires us. Chastity is the
flowering of man; and what are called Genius, Heroism, Holiness, and the like,
are but various fruits which succeed it. Man flows at once to God when the
channel of purity is open.  By turns our purity inspires and our impurity casts
us down. He is blessed who is assured that the animal is dying out in him day by
day, and the divine being established. Perhaps there is none but has cause for
shame on account of the inferior and brutish nature to which he is allied. I
fear that we are such gods or demigods only as fauns and satyrs, the divine
allied to beasts, the creatures of appetite, and that, to some extent, our very
life is our disgrace. ---

\begin{verse}
    \enquote{How happy's he who hath due place assigned \\
        To his beasts and disafforested his mind! \\
        \dots\dots

        Can use this horse, goat, wolf, and ev'ry beast, \\
        And is not ass himself to all the rest! \\
        Else man not only is the herd of swine, \\
        But he's those devils too which did incline \\
        Them to a headlong rage, and made them worse.}\footnote{John Donne
        (1573--1631) \textit{To Sir Edward Herbert}}
\end{verse}

\paragraph{} All sensuality is one, though it takes many forms; all purity is
one. It is the same whether a man eat, or drink, or cohabit, or sleep sensually.
They are but one appetite, and we only need to see a person do any one of these
things to know how great a sensualist he is. The impure can neither stand nor
sit with purity. When the reptile is attacked at one mouth of his burrow, he
shows himself at another. If you would be chaste, you must be temperate. What is
chastity? How shall a man know if he is chaste? He shall not know it. We have
heard of this virtue, but we know not what it is. We speak conformably to the
rumor which we have heard. From exertion come wisdom and purity; from sloth
ignorance and sensuality. In the student sensuality is a sluggish habit of mind.
An unclean person is universally a slothful one, one who sits by a stove, whom
the sun shines on prostrate, who reposes without being fatigued. If you would
avoid uncleanness, and all the sins, work earnestly, though it be at cleaning a
stable. Nature is hard to be overcome, but she must be overcome. What avails it
that you are Christian, if you are not purer than the heathen, if you deny
yourself no more, if you are not more religious? I know of many systems of
religion esteemed heathenish whose precepts fill the reader with shame, and
provoke him to new endeavors, though it be to the performance of rites merely.

\paragraph{} I hesitate to say these things, but it is not because of the
subject --- I care not how obscene my words are --- but because I cannot speak
of them without betraying my impurity. We discourse freely without shame of one
form of sensuality, and are silent about another. We are so degraded that we
cannot speak simply of the necessary functions of human nature. In earlier ages,
in some countries, every function was reverently spoken of and regulated by law.
Nothing was too trivial for the Hindoo lawgiver, however offensive it may be to
modern taste. He teaches how to eat, drink, cohabit, void excrement and urine,
and the like, elevating what is mean, and does not falsely excuse himself by
calling these things trifles.

\paragraph{} Every man is the builder of a temple, called his body, to the god
he worships, after a style purely his own, nor can he get off by hammering
marble instead. We are all sculptors and painters, and our material is our own
flesh and blood and bones. Any nobleness begins at once to refine a man's
features, any meanness or sensuality to imbrute them.

\paragraph{} John Farmer sat at his door one September evening, after a hard
day's work, his mind still running on his labor more or less. Having bathed, he
sat down to recreate his intellectual man. It was a rather cool evening, and
some of his neighbors were apprehending a frost. He had not attended to the
train of his thoughts long when he heard some one playing on a flute, and that
sound harmonized with his mood. Still he thought of his work; but the burden of
his thought was, that though this kept running in his head, and he found himself
planning and contriving it against his will, yet it concerned him very little.
It was no more than the scurf of his skin, which was constantly shuffled off.
But the notes of the flute came home to his ears out of a different sphere from
that he worked in, and suggested work for certain faculties which slumbered in
him. They gently did away with the street, and the village, and the state in
which he lived. A voice said to him --- Why do you stay here and live this mean
moiling life, when a glorious existence is possible for you? Those same stars
twinkle over other fields than these. --- But how to come out of this condition
and actually migrate thither? All that he could think of was to practice some
new austerity, to let his mind descend into his body and redeem it, and treat
himself with ever increasing respect.

\end{document}

